\documentclass{article}

\usepackage{geometry}
\geometry{paperwidth=3in,paperheight=1.7in,margin=0in,bottom=0in,top=0in}
\setlength{\parindent}{0in}
\setlength{\parskip}{0in}
\usepackage{xcolor}
\usepackage{sansmath}
\usepackage{tikz}
\usetikzlibrary{calc}

\begin{document}
\sffamily\sansmath
\thispagestyle{empty}

% x^2 + y^2 + z^2 = 1
% (2/7,3/7,6/7)
% 2x + 3y + 6z = 7

\begin{tikzpicture}[overlay,remember picture,anchor=north west,inner sep=0pt, outer sep=0pt]
\node[xshift=-2cm,yshift=0.25cm] at (current page.south east) {%
 \begin{tikzpicture}[x=1in,y=1in,overlay,remember picture, inner sep=0pt, outer sep=0pt]

\color{black!50!white}

%\draw[->] (-1.15,0) -- (1.15,0);
%\draw[->] (0,-1.15) -- (0,1.15);

\foreach \j in {0.0,0.1,0.2,0.3,0.4,0.5,0.6,0.7,0.8,0.9} {
  \draw[opacity=100,line width=0.25pt] (0,0) circle ({sqrt(1 - (\j)^2)});

  \foreach \i in {0.25,0.50,0.75} {
    \draw[opacity=50,line width=0.1pt] (0,0) circle ({sqrt(1 - (\j + 0.1*\i)^2)});
  }
}

\newcommand{\radius}{0.45}
% 3*y = 7 - 6*z - 2*x
% 2*x = 7 - 6*z - 3*y
\color{red!50!black}
\foreach \j in {0.8,0.9,1.0} {
  \foreach \i in {0.0} {
    \draw[line width=0.25pt] ({2/7 - \radius},{(7 - 6*(\j+0.1*\i) - 2*(2/7 - \radius))/3}) -- ({(7 - 6*(\j+0.1*\i) - 3*(-0.2))/2},{-0.2});
  }

  \foreach \i in {-0.5,-0.4,-0.3,-0.2,-0.1,0.1,0.2,0.3,0.4} {
    \draw[line width=0.1pt] ({2/7 - \radius},{(7 - 6*(\j+0.1*\i) - 2*(2/7 - \radius))/3}) -- ({(7 - 6*(\j+0.1*\i) - 3*(-0.2))/2},{-0.2});
  }
}
  \end{tikzpicture}
};
\end{tikzpicture}

\end{document}

%%% Local Variables: 
%%% mode: latex
%%% TeX-master: t
%%% End: 
