\documentclass{ximera}
\title{Tensor products}

\begin{document}

\begin{abstract}
  Bilinear forms comprise a vector space of tensors.
\end{abstract}	

The set of all bilinear maps from $V \times W \to \R$ has the structure of a vector space: we can add such maps, and multiply them by scalars.

\begin{definition}
  We define $V^* \otimes W^*$ to be the vector space of all bilinear maps from $V \times W \to \R$.

  This is the \textbf{tensor product} of the dual spaces $V^*$ and $W^*$.
\end{definition}
	
Hopefully the reason for the duality involved in the definition above will become clear shortly.
				
Given covectors $S : V \to \R$ and $T : W \to \R$, their \textit{tensor product} is the map 
$S \otimes T: V \times W \to \R$ given by the rule $S \otimes T(\vec{v},\vec{w}) = S(\vec{v}) T(\vec{w})$

\begin{warning}
This formula involves the product of $S(\vec{v})$ and $T(\vec{w})$ \textit{as real numbers.}
\end{warning}

\begin{question}
  $S \otimes T$ is a function from $V \times W$ to $\R$.  Is it bilinear?

  \begin{solution}
    \begin{multiple-choice}
      \choice[correct]{Yes.}
      \choice{No.}
    \end{multiple-choice}
  \end{solution}

  Let's prove it!
  
  \begin{free-response}
    First lets check additivity in the first slot:
    \begin{align*}
      (S \otimes T)(\vec{v_1}+\vec{v_2},\vec{w}) &= S(\vec{v}_1+\vec{v}_2)T(\vec{w})\\
      &= \left(S(\vec{v}_1) +S(\vec{v}_2)\right)T(\vec{w})\\
      &=S(\vec{v}_1)T(\vec{w})+S(\vec{v}_2)T(\vec{w})\\
      &=(S\otimes T)(\vec{v}_1,\vec{w})+(S\otimes T)(\vec{v}_2,\vec{w})
    \end{align*}
    
    Proving additivity in the second slot is similar.
    
    Lets check scaling in the first slot:
    \begin{align*}
      (S \otimes T)(c\vec{v_1},\vec{w}) &= S(c\vec{v})T(\vec{w})\\
      &= cS(\vec{v})T(\vec{w})\\
      &=c(S\otimes T)(\vec{v},\vec{w})
    \end{align*}
    
    Proving scaling in the other slot is similar.
    
    So $S \otimes T$ really is bilinear!
    
  \end{free-response}	
  
\end{question}

	
\end{document}