\documentclass{ximera}

\title{Python}

\begin{document}

\begin{abstract}
  Build up some linear algebra in python.
\end{abstract}

\begin{python}
#We will store a matrix as a list of lists.   For example the list [[1,2],[3,4],[5,6]] will represent the matrix 
# 1 3 5
# 2 4 6
		
#write a function multiply(M_1,M_2) which takes two matrices stored in the above format, and returns the matrix of their product
		
def multiply(M_1,M_2):
  #your code here
		
#Write a function makeFunction(M) which takes a matrix and returns the linear map associated to the matrix.
#For example, makeFunction([[1,2],[3,4]])([2,5]) = [12 32]
		
def makeFunction(M):
  #your code here
		
#Write a function makeMatrix(L) which takes a linear function L ( given as a python function which takes and returns lists) and returns the matrix associated to L.
#For example if  
#def L(v):
#  return [2*v[0]+3*v[1], -4*v[0]]
#Then makeMatrix(L) = BLAH (polymorphism is weird here)
		
def makeMatrix(L):
  #your code 

  #You should double check for some examples that the following is true (as it ought to be)
  # makeMatrix(makeFunction(M_1)(makeFunction(M_2))) = multiply(M_1,M_2)
\end{python}

	


\end{document}
%%% Local Variables: 
%%% mode: latex
%%% TeX-master: t
%%% End: 
