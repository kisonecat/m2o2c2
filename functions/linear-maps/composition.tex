\documentclass{ximera}

\title{Composition}

\begin{document}

\begin{abstract}
  The composition of linear maps can be computed with matrices.
\end{abstract}

Prove that if $S:\R^n \to \R^m$ is a linear map, and $T:\R^m \to \R^k$ is a linear map, then the composite function $T\circ S:\R^n \to R^k$ is also linear.
  
\begin{free-response}
\end{free-response}
	
\begin{question}
  Suppose the matrix of $S$ is $M_S = \begin{bmatrix}
2 & 0 & -1 \\
-1 & 1 & 1
\end{bmatrix}$ and the matrix of $T$ is $M_T = \begin{bmatrix}
-1 & -1 \\
0 & 2 \\
-1 & 1
\end{bmatrix}$.

\begin{solution}
  What is the matrix of $S \circ T$?

  \begin{matrix-answer}[name=M]
    correctMatrix = [['-1','-3'],['0','4']]
  \end{matrix-answer}
\end{solution}

\begin{solution}
  What is the matrix of $T \circ S$?

  \begin{matrix-answer}[name=M]
    correctMatrix = [['-1', '-1', '0'],['-2','2','2'],['-3','1','2']]
  \end{matrix-answer}
\end{solution}

\end{question}
	
\begin{definition}
  If $M$ is a $m\times n$ matrix and $N$ is a $k \times m$ matrix,
  then the \textit{product} $NM$ of the matrices is defined as the
  matrix of the composition of the linear maps defined by $M$ and $N$.

  In other words, $MN$ is the matrix of $L_M\ circ L_N$.
\end{definition}

\begin{warning}
  You may have seen another definition for matrix multiplication in
  the past.  That definition could be seen as a shortcut for how to
  compute the product, but it is usually presented devoid of
  mathematical meaning.

  Hopefully our definition seems properly motivated: \textit{matrix
    multiplication is just what you do to compose linear maps.} We
  suggest working out the problems here using our definition: you will
  develop your own efficient shortcuts in time.
\end{warning}
	
\begin{observation}
It may be easier to think about the following questions if you think about linear maps first, and convert everything over to matrices after.
\end{observation}

\begin{question}
  Suppose $B = \begin{bmatrix} 1 & 2 \\ 3 & 4 \end{bmatrix}$.  Find a $2 \times 2$ matrix $A$ so that $AB \neq BA$.

  \begin{solution}
    \begin{matrix-answer}[name=A]
    function validator(m) {
      if (isWrongSize(m, 2, 2)) return false;
      
      var b = [['1','2'],['3','4']]
      var ab = matrixProduct(m,b);
      var ba = matrixProduct(b,m);
      if (isMatrixCorrect(ab,ba))
        return false;

      return true;
    }
    \end{matrix-answer}  
  \end{solution}
\end{question}
	
\begin{question}
  Find $A \neq 0$ with $AA = 0$.

  \begin{solution}
    \begin{matrix-answer}[name=A]
    function validator(m) {
      if (rows(m) != columns(m)) {
        feedback( 'You should try using a square matrix.' );
        return false;
      }
      
      var m2 = matrixProduct(m,m);
      
      var zeroMatrix = [];
      var i;
      for( i=0; i<rows(m); i++ ) {
        zeroMatrix[i] = [];
        var j;
        for( j=0; j<columns(m); j++ ) {
          zeroMatrix[i][j] = '0';
        }
      }

      if (isMatrixCorrect(m2,zeroMatrix))
        return false;

      return true;
    }
    \end{matrix-answer}
  \end{solution}
\end{question}
	
\begin{question}
  If $A = BLAH$, find $v \neq 0$ with $Av = 0$.
\end{question}
	
\begin{question}
  If $A = BLAH$, find $v$ with $Av = BLAH$.
\end{question}
	
In the last two exercises, you found that solving matrix equations is equivalent to solving systems of linear equations.

\begin{question}
  Rewrite  $System of linear equations$ as $matrix equation$.
\end{question}


\end{document}
%%% Local Variables: 
%%% mode: latex
%%% TeX-master: t
%%% End: 
