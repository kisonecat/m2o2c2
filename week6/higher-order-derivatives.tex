\documentclass{ximera}

\title{Higher order derivatives}

\begin{document}

\begin{abstract}
  Higher derivatives of a function are multilinear maps.
\end{abstract}\maketitle

The $(k+1)^{\text{st}}$ order derivative of a function $f: \R^n \to \R$  at a point $\mathbf{p}$ is a $(k+1)$-linear form  $D^{k+1} f(\mathbf{p})$,
which allows us to approximate changes in the $k^{th}$ order derivative.  This approximation works as follows.

\begin{definition}
  $D^k(\mathbf{p} + \vec{v_{k+1}})(\vec{v_1},\vec{v}_2,\ldots,\vec{v}_k)  = 
  D^k(\mathbf{p})(\vec{v_1},\vec{v}_2,\ldots,\vec{v}_k) + D^{k+1}(\mathbf{p})(\vec{v_1},\vec{v}_2,\ldots,\vec{v}_k,\vec{v}_{k+1})+ 
  \textrm{Error}(\mathbf{p})(\vec{v_1},\vec{v}_2,\ldots,\vec{v}_k,\vec{v}_{k+1})$ where 
  \(
  \displaystyle\lim_{\vec{v}_1,\vec{v}_2,\ldots,\vec{v}_{k+1} \to \vec{0} } 
  \frac{\textrm{Error}(\mathbf{p})(\vec{v_1},\vec{v}_2,\ldots,\vec{v}_k,\vec{v}_{k+1})}{|\vec{v_1}||\vec{v_2}|\cdots|\vec{v_{k+1}}|} = 0
  \)
\end{definition}

\begin{theorem}
  $D^kf(\mathbf{p}) = \sum \frac{\partial^kf}{\partial x_{i_1} \partial x_{i_2} \cdots \partial x_{i_k}} dx_{i_1} \otimes dx_{i_2} \otimes \cdots \otimes dx_{i_k}$, where the sum ranges 
  over all $k$-tuples of basis covectors.
\end{theorem}

\begin{question}
  $f:\R^2 \to \R$ is defined by $f(x,y) = x^2y$.  
  \begin{solution}
    \begin{hint}
      The only terms which are not zero are the terms involving $2$ partial derivatives with respect to $x$ and $1$ partial derivative with respect to $y$.
    \end{hint}
    \begin{hint}
      So $D^3f = \frac{\partial ^3 f}{\partial x \partial x \partial y} dx \otimes dx \otimes dy+ \frac{\partial ^3 f}{\partial x \partial y \partial x} dx \otimes dy \otimes dx+ \frac{\partial ^3 f}{\partial y \partial x \partial y} dy \otimes dx \otimes dx$
    \end{hint}
    \begin{hint}
      So $D^3f(0,0,0) = 2 dx\otimes dx \otimes dy+ 2 dx\otimes dy \otimes dx+2 dy\otimes dx \otimes dx$
    \end{hint}
    \begin{hint}
      So $D^3f(0,0,0)(\verticalvector{1\\2},\verticalvector{3\\4},\verticalvector{0\\1}) =2( 1 \cdot 3 \cdot 1)+2( 1 \cdot 4 \cdot 0)+2(2 \cdot 3 \cdot 0) = 6$
    \end{hint}
    $D^3f(0,0)(\verticalvector{1\\2},\verticalvector{3\\4},\verticalvector{0\\1})=$\answer{$6$}
  \end{solution}
\end{question}

\begin{question}
  Assume \(D^2f(\mathbf{p}) = 3dx_1\otimes dx_2 + 3dx_2 \otimes dx_1\).  In other words, the matrix of $D^f(\mathbf{p})$ is 
  \(\begin{bmatrix} 0 & 3\\ 3&0 \end{bmatrix}\).  Assume $D^3f(\mathbf{p}) = dx \otimes dx \otimes dx$.
  \begin{solution}
    \begin{hint}
      By the definition of higher order derivatives, we have
      
      \(
      D^2f(\mathbf{p}+\verticalvector{0.3\\0.2\\0.3})(\vec{v}_1,\vec{v_2}) \approx D^2f(\mathbf{p})(\vec{v_1},\vec{v_2}) + D^3f(\verticalvector{0.3\\0.2\\0.2},\vec{v_1},\vec{v_2})
      \)
      
    \end{hint}
    \begin{hint}
      So 
      \begin{align*}
        D^2f(\mathbf{p}+\verticalvector{0.3\\0.2\\0.3})(\vec{v}_1,\vec{v_2}) &\approx 
        3dx_1\otimes dx_2(\vec{v}_1,\vec{v}_2) + 3dx_2 \otimes dx_1 (\vec{v}_1,\vec{v_2}) + dx \otimes dx \otimes dx (\verticalvector{0.3\\0.2\\0.3},\vec{v_1},\vec{v_2})\\
        &=3dx_1\otimes dx_2(\vec{v}_1,\vec{v}_2) + 3dx_2 \otimes dx_1 (\vec{v}_1,\vec{v_2})  + 0.3 dx \otimes dx (\vec{v_1},\vec{v_2})
      \end{align*}
    \end{hint}
    \begin{hint}
      The matrix of this bilinear form is \(\begin{bmatrix} 0.3 & 3\\3&0\end{bmatrix}\)
    \end{hint}
    The matrix of the bilinear form $D^2f(\mathbf{p}+\verticalvector{0.3\\0.2\\0.3})$ is approximately
    \begin{matrix-answer}
      correctMatrix = [['0.3','3'],['3','0']]
    \end{matrix-answer}
  \end{solution}
\end{question}


\end{document}
