\documentclass{ximera}

\title{Eigenvectors}

\begin{document}

\begin{abstract}
  Eigenvectors are mapped to multiples of themselves.
\end{abstract}

\begin{definition}
  Let $L:V \to V$ be a linear map.  A vector $\vec{v} \in V$ is called
  an \textbf{eigenvector} of $L$ if $L(\vec{v}) = \lambda \vec{v}$ for
  some $\lambda \in \R$.

  A constant $\lambda \in \R$ is called an \textbf{eigenvalue} of $L$
  if there is a nonzero eigenvector $\vec{v}$ with $L(\vec{v}) =
  \lambda \vec{v}$.
\end{definition}

Geometrically, eigenvectors of $L$ are those vectors whose direction
is not changed (or at worst, negated!) when they are transformed by
$L$.

Let's try some examples.

\begin{question}
  Suppose $L : \R^2 \to \R^2$ is the linear map represented by the matrix
  $$
  \begin{bmatrix}
    3 & 2 \\
    4 & 1
  \end{bmatrix}.
  $$
  
  Which of these vectors is an eigenvector of $L$?
  \begin{solution}
    \begin{hint}
      \begin{question}
        What is $L\left( \begin{bmatrix} 1 \\ -1 \end{bmatrix}\right)$?        

        \begin{solution}
          \begin{hint}
            We want to compute $\begin{bmatrix} 3 & 2 \\ 4 & 1 \end{bmatrix} \begin{bmatrix} 1 \\ -1 \end{bmatrix}$.
          \end{hint}

          \begin{hint}
            In this case, $\begin{bmatrix} 3 & 2 \\ 4 & 1 \end{bmatrix} \begin{bmatrix} 1 \\ -1 \end{bmatrix} = \begin{bmatrix} 1 \\ 3 \end{bmatrix}$.
          \end{hint}

          \begin{matrix-answer}[name=M]
            correctMatrix = [['1'],['3']]
          \end{matrix-answer}
        \end{solution}
        
        Is $\begin{bmatrix} 1 \\ 3 \end{bmatrix}$ a multiple of $\begin{bmatrix} 1 \\ -1 \end{bmatrix}$?

        \begin{solution}
          \begin{multiple-choice}
            \choice[correct]{No.}
            \choice[correct]{Yes.}
          \end{multiple-choice}
        \end{solution}

        Consequently,  $\begin{bmatrix} 1 \\ -1 \end{bmatrix}$ is \textbf{not} an eigenvector.  The eigenvector must be $\begin{bmatrix} 1 \\ 1\end{bmatrix}$.
      \end{question}
    \end{hint}

    \begin{multiple-choice}
      \choice[correct]{$\begin{bmatrix} 1 \\ 1\end{bmatrix}$}
      \choice{$\begin{bmatrix} 1 \\ -1\end{bmatrix}$}
    \end{multiple-choice}
  \end{solution}
  
  That's right!  Note that $L\left( \verticalvector{1 \\ 1} \right)$ is
  $\verticalvector{5 \\ 5} = 5 \cdot \verticalvector{1 \\ 1}$, and so
  $\verticalvector{1 \\ 1}$ is an eigenvector.

  \begin{solution}
    \begin{hint}
      Try computing $L\left( \verticalvector{1 \\ -2} \right)$.
    \end{hint}

    \begin{hint}
      In this case, $L\left( \verticalvector{1 \\ -2} \right) = \verticalvector{-1 \\ 2}$.
    \end{hint}

    \begin{hint}
      \begin{question}
        Find $\lambda \in \R$ so that $\verticalvector{1 \\ -2} = \lambda \cdot \verticalvector{-1 \\ 2}$.
        \begin{solution}
          \begin{hint}
            The sign is opposite on both sides of the equation.
          \end{hint}

          \begin{hint}
            So try $\lambda = -1$.
          \end{hint}

          $\lambda = $ \answer{$-1$}
        \end{solution}

        And so $-1$ is an eigenvalue, with eigenvector $\verticalvector{1 \\ -2}$.
      \end{question}
    \end{hint}

    Which of the following is another eigenvector?
    \begin{multiple-choice}
      \choice[correct]{$\verticalvector{1 \\ -2}$}
      \choice{$\verticalvector{2 \\ -1}$}
    \end{multiple-choice}
  \end{solution}

  Rock on!  We check that 
  $$
  L\left( \verticalvector{1 \\ -2} \right) = \verticalvector{-1 \\ 2} = -1 \cdot \verticalvector{1 \\ -2}
  $$
  and so $\verticalvector{1 \\ -2}$ is an eigenvector with eigenvalue $-1$.
\end{question}


\end{document}