\begin{document}
\section{The Chain Rule}

The chain rule of single variable calculus tells you how the derivative of a composition of functions relates to the derivatives of each of the original functions.

\begin{question}
	Let $f:\R^2 \to R^3$ and $g:\R^3 \to \R^1$.  Given the table of values of $f$, $g$, $Df$ and $Dg$, approximate $g(f(1.01,2.03))$ as well as you can.
	
	BADBAD: 
	Have various values for $f$, including $f(1,2)$
	Have various values for $g$, including $g(f(1,2))$
	Have various values for $Df$, including $Df(1,2)$
	Have various values for $Dg$, including $Dg(f(1,2))$
	\begin{answer}
		$g(f(1.01,2.03)) \approx g(f(1,2)+ Df(1,2)(\verticalvector{0.01,0.03}))
								 \approx g(f(1,2))+ Dg(f(1,2))(Df(1,2)(\verticalvector{0.01,0.03}))$
	\end{answer}
\end{question}

\begin{question}
	Same as previous question, but do not have function values for $g$.  Question is :  what is $D(g \comp f)(1,2)$?
\end{question}

\begin{theorem}
	Let $f:\R^n \to \R^m$ and $g: \R^m \to \R^k$ be differentiable functions, and $\mathbf{p} \in \R^n$.  Then 
	\[
		D(g \comp f)(\mathbf{p}) = Dg(\mathbf{f(p)}) \comp Df(\mathbf{p})
	\]
	
	In other words, the derivative of a composition of functions is the composition of the derivatives of the functions.
\end{theorem}

\begin{question}
	Let us rederive the product rule of single variable calculus using the chain rule of multivariable calculus.
	Let $f,g:\R \to \R$ be differentiable functions.  Let $Multiply:\R^2 \to \R$ be defined by $Multiply(x,y) = xy$.  
	Define $Both:\R \to \R^2$  by $Both(x) = (f(x),g(x))$. 
	Write a formula for $Multiply(Both(x))$.
	What is the jacobian of $D(Multiply)(a,b)$?
	What is the jacobian of $D(Both)(t)$?
	What is the jacobian of $D(Multiply(Both))(t)$? (Use the chain rule!)
	Do you see that this reproves the product rule?
\end{question}

\begin{question}
	Problems where they compute $Df$, $Dg$,  $D(g \comp f)$ via chain rule, $g \comp f$ and then $D(g \comp f)$ again directly.
\end{question}



\end{document}