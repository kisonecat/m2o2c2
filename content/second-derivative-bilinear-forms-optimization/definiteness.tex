\begin{document}
\section{Definiteness of a bilinear form and the Real Spectral Theorem}

\begin{definition}
	A bilinear form $B$ is called 
	\begin{itemize}
		\item \textit{Positive definite} if $B(\vec{v},\vec{v}) > 0 $ for all $\vec{v} \neq \vec{0}$
		\item \textit{Positive semidefinite} if $B(\vec{v},\vec{v}) \geq 0 $ for all $\vec{v}$
		\item \textit{Negative definite} if $B(\vec{v},\vec{v}) < 0 $ for all $\vec{v} \neq \vec{0}$
		\item \textit{Negative semidefinite} if $B(\vec{v},\vec{v}) \leq 0 $ for all $\vec{v}$	
		\item \textit{Indefinite} if $B$ there are $v$ and $w$ with $B(v,v)>0$ and $B(w,w)<0$
		\end{itemize}
\end{definition}

\begin{question}
	What is the definiteness of the bilinear form associated with the matrix $BLAH$?
\end{question}

\begin{question}
	Let $M$ be a diagonal matrix.  Can you relate the sign of the entries $M_{i,i}$ to the definiteness of the associated bilinear form? 
\end{question}

Our goal will now be to reduce the study of general symmetric bilinear forms to those whose associated matrix is diagonal.

\begin{question}
	Let $L: \R^n \to R^n$ be a symmetric linear operator.  Prove that if $\vec{v}_1$ and $\vec{v}_2$ are eigenvectors with 
	distinct eigenvalues $\lambda_1$  and $\lambda_2$, then $\vec{v_1} \perp \vec{v}_2$.
\end{question}

\begin{question}
	Let $L:\R^n \to \R^n$ be a self adjoint linear operator.  Let $\vec{v}$ be an eigenvector of $L$.  Prove that $L$ restricts to a symmetric linear operator
	on the space of vectors perpendicular to $\vec{v}$,  $\vec{v}^\perp$.  
\end{question}

\begin{theorem}
	If $L: \R^n \to \R^n$ is a self adjoint linear operator, then $L$ has an eigenvector.
\end{theorem}

\begin{proof}
	First assume that $L$ is not the identically $0$ map.  If it is, we are done because $0$ is an eigenvector in that case.
	
	Since the unit sphere in $\R^n$ is compact, the function $\vec{v} \mapsto |L(\vec{v})|$ achieves its maximum $M$.  So there is a unit vector 
	$\vec{u}$ so that $|L(\vec{u})| = M$, and $|L(v)| \leq M$ for all other unit vectors $\vec{v}$.  $M > 0$ because WHYWHY
	
	Now let $w = L(v)/M$.  This is another unit vector.
	
	Note that $\langle w, L(v)\rangle = M$, so we also have $\langle L(w), v \rangle = M$, since $L$ is self adjoint.
	
	$\langle L(\vec{w}),\vec{v}\rangle \leq |L(\vec{w})||\vec{v}|$ with equality if and only if $L(\vec{w}) \in span(\vec{v})$ by the WHYWHY theorem.
	
	But $|L(\vec{w})||\vec{v}| = \vec{L(\vec{w})}$ because WHYWHY.
	
	So since $M$ is the maximum value of $|L(\vec{u})|$ over all unit vectors $\vec{u}$, we must have  $L(\vec{w}) \in spam{\vec{v}}$
	
	We can conclude that $L(\vec{w}) = M\vec{v}$.
	
	Now either $\vec{v}+\vec{w} \neq 0$ or $\vec{v}-\vec{w} \neq 0$ .  In either case,
	
	$L(\vec{v} \pm \vec{w}) = M\vec{v} \pm M\vec{w}$, so the nonzero $\vec{v} \pm \vec{w}$ is an eigenvector of $L$.
	
	Credit for this beautiful line of reasoning goes to \href{http://mathoverflow.net/a/118759/1106}{Marcos Cossarini}.  Most proofs of this theorem use either 
	Lagrange Multipliers (which we will learn about soon), or complex analysis.  Here we use only linear algebra along with the one analytic fact that a continuous 
	function on a compact set achieves its maximum value.
\end{proof}

\begin{question}
	Combine the previous theorem and question to prove the ``Spectral Theorem for Real Self-adjoint Operators'':  
	
	\begin{theorem}
		A self adjoint operator $L : \R^n \to \R^n$ has an orthonormal basis of eigenvectors.
	\end{theorem}
	
\end{question}

This, in some sense, completely answers the question of how to characterize the definiteness of a symmetric bilinear form.  Look at its associated linear operator, 
which must be self adjoint.  By the Spectral Theorem, it has a orthonormal basis of eigenvectors.  Then 
\begin{itemize}
	\item $B$ positive definite $\Longleftrightarrow$ $L_B$ has all positive eigenvalues
	\item $B$ positive semidefinite $\Longleftrightarrow$ $L_B$ has all nonnegative eigenvalues
	\item $B$ negative definite $\Longleftrightarrow$ $L_B$ has all negative eigenvalues
	\item $B$ negative semidefinite $\Longleftrightarrow$ $L_B$ has all nonpositive eigenvalues
	\item $B$ indefinite $\Longleftrightarrow$ $L_B$ has both positive and negative eigenvalues
\end{itemize}
	
\end{document}