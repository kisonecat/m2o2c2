\documentclass{ximera}
\title{Jacobian matrix}
\begin{document}

\begin{abstract}
  Code the Jacobian matrix.
\end{abstract}

In the previous activity, we wrote some code to compute \texttt{D}.
Armed with this, we can take a function $f : \R^n \to \R^m$ and a
point $\mathbf{p} \in \R^n$ and compute $Df(p)$, the linear map which
describes, infinitesimally, how wiggling the input to $f$ will affect
its output.

Assuming $f$ is differentiable, we have that $Df(p)$ is a linear map,
so we can write down a matrix for it.  Let's do so now.

\begin{exercise}
To get started, we begin by computing partial derivatives in Python.

To make things easy, let's differentiate functions like 
\begin{verbatim}
def fi(v):
  return v[0] * (v[1]**2)
\end{verbatim}
In other words, our functions will send an $n$-tuple to a single real number.
In that case, \texttt{partial(fi,1)([2,3])} should be 12, since
$$
\frac{\partial}{\partial y} \left( x y^2 \right) = 2 xy,
$$
and so at the point $(2,3)$, the derivative is $2 \cdot 2 \cdot 3 = 12$.

\begin{solution}
  \begin{python}
epsilon = 0.0001
n = 2
#
# fi is a function from R^n to R
def partial(fi,j):
  def derivative(p):
    return # the partial derivative of fi in the j-th coordinate at p
  return derivative

# this should be 12
print partial(lambda v: v[0] * v[1]**2, 1)([2,3])

def validator():
  return abs(partial(lambda v: v[0]**2 * v[1]**3, 0)([7,2]) - 112) < 0.01
  \end{python}
\end{solution}

If we have a function $f : \R^n \to R^m$, we'll encode it as a Python
function which takes a list with $n$ entries, and returns a list with
$m$-entries.  Let's write a Python helper for pulling out just the
$i$th component of the output.

\begin{solution}
  \begin{python}
# if f is a function from R^n to R^m,
#  then component(f,i) is the function R^n to R,
#  which just looks at the i-th entry of the output

def component(f,i):
  return lambda p: # the i-th component of the output

def validator():
  return component(lambda v: [v[0],v[1]],1)([1,17]) == 17
  \end{python}
\end{solution}

Now we put it all together.  For a function $f : \R^n \to \R^m$, the \textbf{Jacobian matrix} is given by
\[
\begin{bmatrix}
  \frac{\partial f_1}{\partial x_1} \left(\mathbf{p}\right) & \frac{\partial f_1}{\partial x_2} \left(\mathbf{p}\right) & \cdots & \frac{\partial f_1}{\partial x_n}\left(\mathbf{p}\right) \\
  \frac{\partial f_2}{\partial x_1} \left(\mathbf{p}\right) & \frac{\partial f_2}{\partial x_2} \left(\mathbf{p}\right) & \cdots & \frac{\partial f_2}{\partial x_n}\left(\mathbf{p}\right) \\
  \vdots                                                    & \vdots                                                    & \ddots & \vdots \\
  \frac{\partial f_m}{\partial x_1} \left(\mathbf{p}\right) & \frac{\partial f_m}{\partial x_2} \left(\mathbf{p}\right) & \cdots & \frac{\partial f_m}{\partial x_n}\left(\mathbf{p}\right) 
\end{bmatrix}
\]

\begin{solution}
  \begin{python}
epsilon = 0.0001
n = 3 # the dimension of the domain
m = 2 # the dimension of the codomain
def component(f,i):
  return lambda p: f(p)[i]
def partial(fi,j):
  def derivative(p):
    p_shifted = p
    p_shifted[j] = p[j] + epsilon
    return (fi(p_shifted) - fi(p))/epsilon
  return derivative
#
# f is a function from R^n to R^m
# jacobian(f) is its Jacobian matrix
def jacobian(f):
  return # the Jacobian matrix

def validator():
  return component(lambda v: [v[0],v[1]],1)([1,17]) == 17
  \end{python}
\end{solution}

\end{exercise}

\end{document}
