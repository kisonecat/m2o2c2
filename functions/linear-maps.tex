\documentclass{ximera}

\title{Linear maps}

\begin{document}

\begin{abstract}
  Linear maps respect addition and scalar multiplication.
\end{abstract}

We begin by defining linear maps.

\begin{definition}
  A function $L: \R^n \to \R^m$ is called a \textit{linear map} if it
  ``respects addition and scalar multiplication.''

  Symbolically, for a map to be linear, we must have that $L(v+w) =
  L(v)+L(w)$ for all $v,w \in \R^n$ and also $L(av) = a L(v)$ for all
  $a \in \R$ and $v\in \R^n$.
\end{definition}

\begin{question}
  Which of the following functions are linear?
  \begin{solution}
    \begin{hint}
    	For a function to be linear, it must respect scalar multiplication.  Lets see what how $f(5 \vecticalvector{1\\1})$ compares to $f(\verticalvector{1\\1})$, and also how
	$h(5 \vecticalvector{1\\1})$ compares to $h(\verticalvector{1\\1})$.  
	
	\begin{question}
		\begin{solution}
		 What is $f(5 \verticalvector{1\\1})$?
		 \answer{15}
		\end{solution}
		\begin{solution}
		 What is $f(\verticalvector{1\\1})$?
		 \answer{3}
		\end{solution}
		\begin{solution}
			Is $f(5 \vecticalvector{1\\1}) = 5 f(\verticalvector{1\\1})$?
			\begin{multiple-choice}
			\choice[correct]{Yes}
			\choice{No}
			\end{multiple-choice}
		\end{solution}
		Great!  So $f$ has a chance of being linear, since it is respecting scalar multiplication in this case.
		What about $h$?
		begin{solution}
		 What is $h(5 \verticalvector{1\\1})$?
		\begin{matrix-answer}[name=v]
    			  correctMatrix = [['17'],['5']]
   		 \end{matrix-answer}
		\end{solution}
		\begin{solution}
		 What is $h(\verticalvector{1\\1})$?
		 \begin{matrix-answer}[name=v]
    			  correctMatrix = [['17'],['1']]
   		 \end{matrix-answer}
		\end{solution}
		\begin{solution}
			Is $f(5 \vecticalvector{1\\1}) = 5 f(\verticalvector{1\\1})$?
			\begin{multiple-choice}
			\choice{Yes}
			\choice[correct]{No}
			\end{multiple-choice}
		\end{solution}
		Great!  So $h$ is not linear:  by looking at this particular example, we can see that $h$ does not always respect scalar multiplication.  So $h$ is not linear.
		
		Since we know one of the two functions is linear, we can already answer the question:  The answer is $f$.  To be through, lets check that $f$ really is linear.
		
		First we check that $f$ really does respect scalar multiplication:
		
		Let $a \in \R$ be an arbitrary scalar and $\verticalvector{x\\y} \in \R^2$ be an arbitrary vector.  Then
		
		\begin{align*}
		 f(a\verticalvector{x\\y} \in \R^2) &= f(\verticalvector{ax\\ay})\\
		 &= ax+2ay\\
		 &= a(x+2y)\\
		 &=af(\verticalvector{x\\y}) 		
		 \end{align*}
		 
		 Now we check that $f$ really does respect vector addition:
		 
		 Let $\verticalvector{x_1\\y_1}$ and $\verticalvector{x_2\\y_2}$ be arbitrary vectors in $\R^2$.  Thn
		 
		 \begin{align*}
		 f(\verticalvector{x_1\\y_1}+\verticalvector{x_2\\y_2})&= f(\verticalvector{x_1+x_2\\y_1+y_2})\\
		 &= (x_1+x_2)+2(y_1+y_2)\\
		 &= x_1+x_2+2y_1+2y_2\\
		 &=(x_1+2y_1)+(x_2+2y_2)\\
		 &=f(\verticalvector{x_1\\y_1})+f(\verticalvector{x_2,y_2})
		 \end{align*}
		 
		 This proves that $f$ is linear!
		
	\end{question}
	
    \end{hint}
    \begin{multiple-choice}
      \choice[correct]{\(f: \R^2 \to \R^1\) defined by \(f\left(\verticalvector{x \\ y}\right) = x+2y\)}

      \choice{$h:\R^2 \to \R^2$ defined by $h\left( \verticalvector{x \\y} \right) = \verticalvector{17 \\ x}$}
    \end{multiple-choice}
  \end{solution}
\end{question}
% BADBAD: include hints

\begin{question}
  Which of the following functions are linear?
  \begin{solution}
    \begin{multiple-choice}
    \choice{$g: \R^3 \to R^2$ defined by $g\left(\verticalvector{x\\y\\z}\right) = \verticalvector{x\\xy}$}
    \choice[correct]{$h:\R \to \R^4$ defined by $h(x) = \verticalvector{x\\x\\x\\4x}$}
    \end{multiple-choice}
  \end{solution}
\end{question}
% BADBAD: include hints

\begin{question}
  Which of the following functions are linear?
  \begin{solution}
    \begin{multiple-choice}
      \choice{$G: \R^4 \to  \R^3$ defined by $G\left(\verticalvector{x\\y\\z\\t}\right) = \verticalvector{e^{x+y}\\x+y\\ \sin(x+y)}$}
      \choice{$A: \R^2 \to R^2$ defined by $A\left(\verticalvector{x\\y}\right)=\verticalvector{0\\0}$}
    \end{multiple-choice}
  \end{solution}

  The function which sends every vector to the zero vector \textit{is} linear.
\end{question}
% BADBAD: include hints
	
\begin{question}
  Let $L:\R^3 \to \R^2$ be a linear function.  Suppose
  $L\left(\verticalvector{1\\0\\0}\right) = \verticalvector{3\\4}$,
  $L\left(\verticalvector{0\\1\\0}\right) = \verticalvector{-2\\0}$,
  and $L\left(\verticalvector{0\\0\\1}\right) =
  \verticalvector{1\\-1}$.
  
  \begin{solution}
    Let $\vec{v} = L \left(\verticalvector{4\\-1\\2}\right)$.  What is $\vec{v}$?

    \begin{matrix-answer}[name=v]
      correctMatrix = [['16'],['14']]
    \end{matrix-answer}
    
  \end{solution}

  Can you generalize this?

  \begin{solution}
    Let $\vec{v} = L\left(\verticalvector{x\\y\\z}\right)$?  What is $\vec{v}$?

    \begin{matrix-answer}[name=v]
      correctMatrix = [['3*x - 2*y + z'],['4*x - z']]
    \end{matrix-answer}
    
  \end{solution}

  As you have already discovered a linear map $L: \R^n \to \R^m$ is
  fully determined by its action on the ``standard basis vectors''
  $e_1 = \verticalvector{1\\0\\0\\\vdots\\0}$, $e_2 =
  \verticalvector{0\\1\\0\\\vdots\\0}$, and so on, until we reach $e_n
  = \verticalvector{0\\0\\\vdots\\0\\1}$.

  Argue convincingly that if $L:\R^n \to\R^m$ is a linear map and you know $L(\vec{e}_i)$ for $i=1,2,3,...,n$, then you could figure out $L(\vec{v})$ for
  any $\vec{v} \in \R^n$.
  \begin{free-response}
    % BADBAD: include solution
  \end{free-response}
\end{question}

\end{document}
