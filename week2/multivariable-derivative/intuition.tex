\documentclass{ximera}
\title{Intuitively}

\begin{document}
\begin{abstract}
  The derivative is the linear map which best approximates changes in a function near a point.
\end{abstract}

The single variable derivative allows us to find the best linear
approximation to a function at a point.  In several variables we will
\textit{define} the derivative to be a linear approximation which
approximates the change in the values of a function.  In this section
we will explore what the multivariable derivative is from an intuitive
point of view, without making anything too formal.

We give the following wishy-washy ``definition'':

\begin{definition}
	Let $f:\R^n \to \R^m$ be a function.  Then the derivative of $f$ at a point $\mathbf{p} \in \R^n$ is the linear map $D(f)\big|_{\mathbf{p}}:\R^n \to \R^m$ 
	which allows the following approximation property:
	
	\[
		f(\mathbf{p} + \vec{h}) \approx f(x) + D(f)\big|_{\mathbf{p}}(\vec{h})
	\]
	
	We will make the sense in which this approximation holds precise in the next section.
\end{definition}

Note:  we also call the matrix of the derivative the \textbf{Jacobian Matrix} in honor of the mathematician 
\href{http://en.wikipedia.org/wiki/Carl_Gustav_Jacob_Jacobi}{Carl Gustav Jacob Jacobi}.

\begin{question}
	Let $f:\R^2 \to \R^3$ be a function, and suppose $f(2,3) = (4,8,9)$.  Suppose that the matrix of $D(f)\big|_{(2,3)}$ is 
	$\begin{bmatrix} -1 & 3\\ 4& 5 \\ 2 & -3 \end{bmatrix}$.  
	
	\begin{solution}
		\begin{hint}
			By the defining property of derivatives,
			
			$f(2.01,3.04) \approx f(2,3) + D(f)\big|_{(2,3)}\left(\verticalvector{0.01\\0.04}\right)$
		\end{hint}
		\begin{hint}
			 $= \verticalvector{4\\8\\9}+\begin{bmatrix} -1 & 3\\ 4& 5 \\ 2 & -3 \end{bmatrix}\verticalvector{0.01\\0.04}$
		\end{hint}
		\begin{hint}
			 $= \verticalvector{4\\8\\9}+\verticalvector{0.01(-1)+0.04(3)\\0.01(4)+0.04(5)\\0.01(2)+0.04(-3)}$
		\end{hint}
		\begin{hint}
			 $= \verticalvector{4\\8\\9}+\verticalvector{0.11\\0.24\\-0.1}$
		\end{hint}
		\begin{hint}
			 $= \verticalvector{4.11\\8.24\\8.9}$
		\end{hint}
		Approximate $f(2.01,3.04)$, giving your answer as a column matrix.
		\begin{matrix-answer}[name = v]
			correctMatrix = [['4.11'],['8.24'],['8.9']]
		\end{matrix-answer}
	\end{solution}
\end{question}

\begin{question} 
Let $f:\R^2 \to \R$ be a function with $f(1,2) = 3$, $f(1.01,2) = 3.04$ and $f(1,2.002) = 3.002$.  

\begin{solution}
	\begin{hint}	
		Since $f:\R^2 \to \R$, $D(f)\big|_{(1,2)}: \R^2 \to \R$, so the matrix of the derivative is a row of length $2$.
	\end{hint}
	\begin{hint}
		To find the matrix, we need to see how $ D(f)\big|_{(1,2)}$ acts on $\verticalvector{1\\0}$ and $\verticalvector{0\\1}$
	\end{hint}
	\begin{hint}
		$f(1.01,2) \approx f(1,2) + D(f)\big|_{(1,2)}\left( \begin{bmatrix}0.01\\0\end{bmatrix} \right)$ by the fundamental property of the derivative
	\end{hint}
	\begin{hint}
		\begin{align*}
		& 3.04  \approx 3+ D(f)\big|_{(1,2)}\left( \verticalvector{0.01\\0} \right) \\
		& 0.04  \approx 0.01D(f)\big|_{(1,2)}\left( \\verticalvector{1\\0} \right) \text{ by the linearity of the derivative}\\
		& D(f)\big|_{(1,2)}\left( \verticalvector{1\\0} \right) \approx 4
		\end{align*}
	\end{hint}
	\begin{hint}
		\begin{align*}
		&f(1,2.002) \approx f(1,2) + D(f)\big|_{(1,2)}\left( \verticalvector{0\\0.02} \right)\\
		&3.002 \approx 3+ D(f)\big|_{(1,2)}\left( \verticalvector{0\\0.02} \right)\\
		&0.002 \approx 0.002D(f)\big|_{(1,2)}\left( \verticalvector{0\\1} \right) \text{by the linearity of the derivative}\\
		&D(f)\big|_{(1,2)}\left( \verticalvector{0\\1} \right) \approx 1
		\end{align*}
	\end{hint}
	\begin{hint}
		Thus the matrix of $D(f)\big|_{(1,2)}$ is $\begin{bmatrix} 4 & 1\end{bmatrix}$
	\end{hint}
	
	What is Jacobian matrix of $f$ at $(1,2)$?
	\begin{matrix-answer}[name = M]
		correctMatrix = [['4', '1']]
	\end{matrix-answer}
\end{solution}

\begin{solution} 
	\begin{hint}
		$f(0.9,2.03) \approx f(1,2) +D(f)\big|_{(1,2)}\left(\verticalvector{-0.1\\0.03}\right)$
	\end{hint}
	\begin{hint}
		$ = 3 + \begin{bmatrix} 4 & 1\end{bmatrix}\verticalvector{-0.1\\0.03}$
	\end{hint}
	\begin{hint}
		$ = 3+4(-0.1)+1(0.03)$
		
		$= 2.63$
	\end{hint}
	Using your approximation of the  Jacobian matrix, $f(0.9,2.03) \approx $ \answer{$2.63$}
\end{solution}
This problem shows that if a function has a derivative, then only knowing how it changes in the coordinate directions lets you determine how it changes in \textbf{any} 
direction.  This is so important it is worth driving it home:  we only started with information about how $f(1.01,2)$ and $f(1,2.02)$ compared to $f(1,2)$, but because
this function had a derivative, we could obtain the approximate value of the function at \textbf{any} near by point by exploiting linearity.  \textbf{This is powerful.}
\end{question}

\textbf{Prepare yourself}:  the following two paragraphs are going to be very difficult to digest.

So far we have only talked about the derivative of a function at a point.  The derivative of a function is actually a function which assigns a linear map
to each point in the domain of the original function.   So the derivative is a function which takes (functions from $\R^n \to \R^m$ ) and 
returns  a function which takes ( points in $\R^n$) and  returns ( linear maps from $\R^n \to \R^m$).    This level of abstraction is why we 
wanted you to get comfortable with ``higher-order functions'' earlier.  We are not as crazy as we seem.

As an example, if $f:\R^2 \to \R^2$ is the function defined by $f(x,y) = (x^2y,y+x)$, then it will turn out that at any point $(a,b)$,
 the derivative $Df\big|_{(a,b)}$ will be the linear map from $\R^2$ to $\R^3$ given by the matrix 
 $\begin{bmatrix}2ab & 2a^2\\ 1 & 1\end{bmatrix}$ (we do not know why yet, but this is true).  So $Df$ is really a 
 function which takes a point $(a,b)$ and spits out the linear map with matrix $\begin{bmatrix}2ab & 2a^2\\ 1 & 1\end{bmatrix}$. 
  So what about just plain old $D$?  $D$ takes a function ($f$) and returns the function $Df$ which takes a point $(a,b)$ and returns the linear map
  whose matrix is $\begin{bmatrix}2ab & 2a^2\\ 1 & 1\end{bmatrix}$.  Letting $L(A,B)$ stand for all the linear functions from 
  $A\to B$ and $\text{Func}(A,B)$ be the set of all functions from $A \to B$, we could write $D: \text{Func}(\R^n,\R^m) \to \text{Func}(\R^n,L(\R^n,\R^m))$.
  
  Please do not give up on the course after the last two paragraphs!  Everything is going to be okay.  Hopefully you will be able to slowly digest these statements 
  throughout the course.  Not understanding them now will not hold you back.

\begin{question} 
	Let $f$ be a function which satisfies $Df\big|_{(x,y,z)} = \begin{bmatrix} 3x^2y^2 & 2x^3y\\ 2x&2y\\ye^{xy} & xe^{xy}\end{bmatrix}$.
	
		\begin{solution}
			\begin{hint}
				\(D(f)\big|_{(1,2)} = \begin{bmatrix} 3(1)^2(2)^2 & 2(1)^3(2)\\ 2(1) & 2(2) \\ 2e^{(1)(2)} & 1e^{(1)(2)}\end{bmatrix}\)
			\end{hint}
			\begin{hint}
				\( f\left((1,2)+ \verticalvector{0.01\\-0.02}\right)\approx f(1,2) + 
				\begin{bmatrix} 12 & 4\\2 & 4 \\ 2e^{2} & e^{2}\end{bmatrix} \verticalvector{0.01\\-0.02}\)
			\end{hint}
			\begin{hint}
				\( f(1.01,1.99) \approx (2,3,1)+ \verticalvector{12(0.01)+4(-0.04)\\ 2(0.01)+4(-0.02)\\2e^2(0.01)+e^2(-0.02)}\)
			\end{hint}
			\begin{hint}
				\( f(1.01,1.99) \approx (2.08,2.94,1)\)
			\end{hint}
			\begin{hint}
				Format this as $\verticalvector{2.08\\2.94\\1}$
			\end{hint}
			Given that \(f(1,2) = (2,3,1)\).  Approximate \(f(1.01,1.98)\).  
			\begin{matrix-answer}
				correctMatrix = [['2.08'],['2.94'],['1']]
			\end{matrix-answer}
		\end{solution}
\end{question}




	
\end{document}
