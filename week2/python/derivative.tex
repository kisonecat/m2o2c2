\documentclass{ximera}
\title{Derivative}
\begin{document}

\begin{abstract}
  Code the total derivative.
\end{abstract}

\begin{exercise}
  Let \texttt{epsilon} be a small, but positive number.  Suppose $f:\R
  \to \R$ has been coded as a Python function \texttt{f} which takes a
  real number and returns a real number.  Seeing as
  $$
  f'(x) = \lim_{h \to 0} \frac{f(x+h) - f(x)}{h},
  $$
  can you find a Python function which approximates $f'(x)$?

  Given a Python function \texttt{f} which takes a real number and
  returns a real number, we can approximate $f'(x)$ by using
  \texttt{epsilon}.  Write a Python function \texttt{derivative} which
  takes a function \texttt{f} and returns an approximation to its
  derivative.

\begin{solution}
  \begin{hint}
    To approximate this, use \texttt{(f(x+epsilon) - f(x))/epsilon}.
  \end{hint}
\begin{python}
epsilon = 0.0001
def derivative(f):
  def df(x): return (f(blah blah) - f(blah blah)) / blah blah
  return df


def validator():
  df = derivative(lambda x: 1+x**2+x**3)
  if abs(df(2) - 16) > 0.01:
    return False
  df = derivative(lambda x: (1+x)**4)
  if abs(df(-2.642) - -17.708405152) > 0.01:
    return False
  return True
\end{python}
\end{solution}

  Great work!  Now let's do this in a multivariable setting.

  A function $f:\R^n \to \R^m$ should be stored as a Python function
  which takes a list with $n$ entries and returns a list with $m$
  entries.  
  \begin{solution}
    Implement $f(x,y,z) = (xy,x+z)$ as a Python function.
    \begin{hint}
You can get away with
\begin{verbatim}
def f(v):
  return [v[0]*v[1],v[0] + v[2]]
\end{verbatim}
    \end{hint}
    \begin{python}
def f(v):
  x = v[0]
  y = v[1]
  z = v[2]
  return # such and such

def validator():
  if f([3,2,7])[0] != 6:
    return False
  if f([3,2,7])[1] != 10:
    return False
  return True
    \end{python}
  \end{solution}

  Now we provide you with a function \texttt{add\_vector} which takes two
  vectors $\vec{v}$ and $\vec{w}$ and returns $\vec{v}+\vec{w}$, and a
  function \texttt{scale\_vector} which takes a scalar $c$ and a
  vector $\vec{v}$ and returns the vector $c\vec{v}$.  Finally,
  \texttt{vector\_length(v)} computes the length of the vector
  $\vec{v}$.

  Given all of this preamble, write a function \texttt{D} which takes
  a Python function $f: \R^n \to \R^m$ and returns the function $Df:
  \R^n \to Lin(\R^n,\R^m)$ which takes a point $\mathbf{p} \in \R^n$
  and returns (an approximation to) the linear map $Df(p):\R^n \to
  \R^m$.
	
\begin{solution}
  \begin{hint}
\begin{verbatim}
def D(ff):
  def Df(p):
    f = ff # band-aid over a Python interpreter bug
    def L(v):
      return scale_vector( 1/(epsilon),
                           add_vector( f(add_vector(p, scale_vector(epsilon, v))),
                                       scale_vector(-1, f(p)) ) )
    return L
  return Df
\end{verbatim}
  \end{hint}
\begin{python}
epsilon = 0.0001
n = 3
m = 2
def add_vector(v,w):
  return [sum(v) for v in zip(v,w)]
def scale_vector(c,v):
  return [c*x for x in v]
def vector_length(v):
  return sum([x**2 for x in v])**0.5
def D(f):
  def Df(p):
    f = f # band-aid over a Python interpreter bug
    def L(v):
      # Try "f(p + blah blah) - f(p)" and so on...
      return L(v) where L = Df(p)
    return L
  return Df


def validator():
  # f(x,y,z) = (3*x^2 + 2*x*y*z, x*y^3*z^2)
  Df = D(lambda v: [3*v[0]*v[0] + 2*v[0]*v[1]*v[2], v[0]*(v[1]**3)*(v[2]**2)])
  Dfp = Df([2,3,4])
  Dfpv = Dfp([3,2,1])
  if abs(Dfpv[0] - 152) > 1.0:
    return False
  if abs(Dfpv[1] - 3456) > 10.0:
    return False
  return True
\end{python}
\end{solution}

Note that \texttt{Df(p)} is a linear map, so we can represent that linear map as a matrix.  We do so in the next activity.
\end{exercise}

\end{document}
