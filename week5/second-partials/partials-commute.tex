\documentclass{ximera}
\title{Mixed partials commute}

\begin{document}
	\begin{abstract}
		Order of partial differentiation doesn't matter
	\end{abstract}
	
	In the last section on partial derivatives we made the interesting observation that $\frac{\partial^2 f}{\partial x_i \partial x_j} = \frac{\partial^2 f}{\partial x_j \partial x_i}$
	for all of the functions we considered.  We will now prove this, modulo some technical assumptions.
	
	\begin{theorem}
		Let $f:\R^n \to \R$ be a differentiable function.  Assume that the partial derivatives $f_{x_i}:\R^n \to \R$ are all differentiable, and the second partial derivatives
		$f_{x_i,x_j}$ are continuous.  Then $f_{x_i,x_j} = f_{x_j,x_i}$.
	\end{theorem}
	
	First, let's develop some intuition about why this result is true.  This informal discussion will also suggest how we should proceed with the formal proof.
	
	Let's restrict our attention, for the moment, to functions $g: \R^2 \to \R$.  Observe that  $g_{x}(a,b) \approx \frac{g(a+h,b)-g(a,b)}{h}$ for small values of $h$.  Analogously,
	$g_{y}(a,b) \approx \frac{g(a,b+k)-g(a,b)}{k}$.
	
	Now applying this idea twice, we have 
	
	\begin{align*}
		f_{xy}(a,b) &\approx \frac{1}{h} \left( f_y(a+h,b) - f_y(a,b)\right)\\
			&\approx \frac{1}{h} \left( \frac{f(a+h,b+k)-f(a+h,b)}{k} - \frac{f(a,b+k)-f(a,b)}{k}\right)\\
			&=\frac{f(a+h,b+k)-f(a+h,b)-f(a,b+k)+f(a,b)}{hk}
	\end{align*}
	
	Going through the same process with $f_{yx}$ leads to exactly the same approximation!
	
	So our strategy of proof will be to show that we can express both of these partial derivatives as the two variable limit:
	
	\[
		f_{xy}(a,b) = \displaystyle\lim_{h,k \to 0} \frac{f(a+h,b+k)-f(a+h,b)-f(a,b+k)+f(a,b)}{hk} = f_{yx}(a,b)
	\]
	
	\begin{proof}
		Let $\textrm{HODQ}(h,k) = f(a+h,b+k)-f(a+h,b)-f(a,b+k)+f(a,b)$.  (Here $\textrm{HODQ}$ stands for "higher order difference quotient").
		
		Let $Q(s) = f(s,b+k) - f(s,b)$.
		
		Then $\textrm{HODQ}(h,k) = Q(a+h) - Q(a)$.
		
		By the mean value theorem for derivatives, there is an $0<\epsilon_1 < h$ such that 
		$Q(a+h)-Q(a) = hQ'(a+\epsilon_1)$.
		
		So $\textrm{HODQ}(h,k) = h( f_{x}(a+\epsilon_1,b+k) -f_x(a,b))$ .
		
		By the mean value theorem again, we have
		
		$\textrm{HODQ}(h,k) = hkf_{yx}(a+\epsilon_1,b+\epsilon_2)$ for some $0<\epsilon_2<k$.
		
		Now apply exactly the same reasoning to conclude that $\textrm{HODQ}(h,k) = hkf_{yx}(a+\xi_2,b+\xi_1)$ for some $0<\xi_1<k$ and $0<\xi_2<h$.
		\begin{free-response}
		
		Let $R(s) = f(a+h,s) - f(a,s)$.
		
		Then $\textrm{HODQ}(h,k) = R(b+k) - R(b)$.
		
		By the mean value theorem for derivatives, there is an $0<\xi_1 < k$ such that 
		$ R(b+k) - R(b) = kR'(b+\xi_1)$.
		
		So $\textrm{HODQ}(h,k) = k( f_y(a+h,b+\xi_1) -f_y(a,b))$ .
		
		By the mean value theorem again, we have
		
		$\textrm{HODQ}(h,k) = hkf_{xy}(a+\xi_2,b+\xi_1)$ for some $0<\xi_2<h$.
			
		\end{free-response}
		
		So we have 
		\begin{align*}
			\displaystyle\lim_{h,k \to 0} \frac{f(a+h,b+k)-f(a+h,b)-f(a,b+k)+f(a,b)}{hk} &= \displaystyle\lim_{h,k \to 0} \frac{\textrm{HODQ}(h,k)}{hk}\\
				&= \displaystyle\lim_{h,k \to 0} f_{yx}( a+\epsilon_1,b+\epsilon_2) 
		\end{align*}
		
		But since $0<\epsilon_1 < h$ and $0<\epsilon_2<k$ , then as $h,k \to 0$, $a+\epsilon_1 \to a$ and $b+\epsilon_2 \to b$.  By the continuity of $f_{yx}$, we have that 
		the limit equals $f_{yx}(a,b)$.
		
		Apply the same reasoning to conclude that $\displaystyle\lim_{h,k \to 0} \frac{f(a+h,b+k)-f(a+h,b)-f(a,b+k)+f(a,b)}{hk} = f_{xy}(a,b)$
		\begin{free-response}
		\begin{align*}
			\displaystyle\lim_{h,k \to 0} \frac{f(a+h,b+k)-f(a+h,b)-f(a,b+k)+f(a,b)}{hk} &= \displaystyle\lim_{h,k \to 0} \frac{\textrm{HODQ}(h,k)}{hk}\\
				&= \displaystyle\lim_{h,k \to 0} f_{xy}( a+\xi_2,b+\xi_1) 
		\end{align*}
		
		But since $0<\xi_1 < k$ and $0<\xi_2<h$ , then as $h,k \to 0$, $a+\xi_2 \to a$ and $b+\xi_1 \to b$.  By the continuity of $f_{xy}$, we have that 
		the limit equals $f_{xy}(a,b)$.
		
		\end{free-response}
		
		So we can conclude that $f_{xy}(a,b) = f_{yx}(a,b)$, because they are the common value of the limit $\displaystyle\lim_{h,k \to 0} \frac{f(a+h,b+k)-f(a+h,b)-f(a,b+k)+f(a,b)}{hk}$.
	\end{proof}
	
	We close with a cautionary example.  This result is not always true if the second partial derivatives are not continuous.  Remember that we define
	
	$g_x(a,b) = \displaystyle\lim_{h \to 0} \frac{g(a+h,b)-g(a,b)}{h}$, and similarly $g_y(a,b) = \displaystyle\lim_{k \to 0} \frac{g(a,b+k)-g(a,b)}{k}$
	
	\begin{question}
		Define \(f(x,y) = \begin{cases} 
				xy\frac{x^2-y^2}{x^2+y^2} \text{ if $(x,y) \neq (0,0)$}\\
				0 \text{ if (x,y)=(0,0)}
				\end{cases}\)
		\begin{solution}
			
			\begin{hint}
				\begin{question}
					\begin{solution}
						\begin{hint}
							\begin{align*}
							f_y(x,0) &=  \displaystyle\lim_{k \to 0} \frac{f(x,k)-f(x,0)}{k}\\
								&= \displaystyle\lim_{k \to 0} \frac{xk\frac{x^2-k^2}{x^2+k^2} - 0}{k}\\
								&=\displaystyle\lim_{k \to 0} x\frac{x^2-k^2}{x^2+k^2}\\
								&=x
							\end{align*}
						\end{hint}
						$f_y(x,0) = $\answer{$x$}
					\end{solution}
					\begin{solution}
						\begin{hint}
							\begin{align*}
							f_x(0,y) &=  \displaystyle\lim_{h \to 0} \frac{f(h,y)-f(0,y)}{h}\\
								&= \displaystyle\lim_{h \to 0} \frac{hy\frac{h^2-y^2}{h^2+y^2} - 0}{h}\\
								&=\displaystyle\lim_{h \to 0} y\frac{h^2-y^2}{h^2+y^2}\\
								&=-y
							\end{align*}
						\end{hint}
						$f_x(0,y) = $\answer{$-y$}
					\end{solution}
				\end{question}
			\end{hint}
			\begin{hint}
				\begin{align*}
					f_{xy}(0,0) &= \displaystyle\lim_{h \to 0} \frac{f_y(h,0) - f_y(0,0)}{h}\\
						&=\displaystyle\lim_{h \to 0} \frac{h -0}{h}\\
						&=1
				\end{align*}
			\end{hint}
			\begin{hint}
				\begin{align*}
					f_{yx}(0,0) &= \displaystyle\lim_{k \to 0} \frac{f_x(0,k) - f_x(0,0)}{k}\\
						&=\displaystyle\lim_{h \to 0} \frac{-k -0}{k}\\
						&=-1
				\end{align*}
			\end{hint}
			
			$f_{xy}(0,0)=$\answer{$1$}
		\end{solution}
		\begin{solution}
			$f_{yx}(0,0)=$\answer{$-1$}
		\end{solution}
	\end{question}
	
	
	
\end{document}