\documentclass{article}

\begin{document}

\begin{abstract}
  The composition of linear maps can be computed with matrices.
\end{abstract}

\begin{question}
  Prove that if $S:\R^n \to \R^m$ is a linear map, and $T:\R^m \to \R^k$ is a linear map, then the composite function $T\circ S:\R^n \to R^k$ is also linear.
\end{question}

\begin{question}
  If the matrix of $S$ is $M_S = BLAH$ and the matrix of $T$ is $M_T = BLAH$, find the matrix of $T \circ S$.
\end{question}


\begin{definition}
  If $M$ is a $m\times n$ matrix and $N$ is a $k \times m$ matrix, then the \textit{product} $NM$ of the matrices is
  defined as the matrix of the composition of the linear maps defined $M$ and $N$.  In other words  $MN$ is the matrix of 
  $L_M\ circ L_N$.
\end{definition}

WARNING:  you may have seen another definition for matrix multiplication in the past.  That definition could be seen as a shortcut for how
to compute the product, but it is usually presented devoid of mathematical meaning.  Hopefully our definition seems properly motivated:  matrix multiplication is 
just what you do to compose linear maps.  We suggest working out the problems here using our definition:  you will develop your own efficient shortcuts in time.

\begin{question}
  If $M = BLAH$ and $N=BLAH$, compute $NM$.
\end{question}

Note:  it may be easier to think about the following questions if you think about linear maps first, and convert everything over to matrices after.

\begin{question}
  Find $2\times 2$ matrices $A$ and $B$ with $AB \neq BA$.
\end{question}

\begin{question}
  Find $A \neq 0$  with $AA = 0$.
\end{question}

\begin{question}
  If $A = BLAH$, find $v \neq 0$ with $Av = 0$ 
\end{question}

\begin{question}
  If $A = BLAH$, find $v$ with $Av = BLAH$ 
\end{question}

In the last two exercises, you found that solving matrix equations is equivalent to solving systems of linear equations.

\begin{question}
  Rewrite  $System of linear equations$ as $matrix equation$.
\end{question}

\begin{python}
  #We will store a matrix as a list of lists.   For example the list [[1,2],[3,4],[5,6]] will represent the matrix 
  # 1 3 5
  # 2 4 6
  
  #write a function multiply(M_1,M_2) which takes two matrices stored in the above format, and returns the matrix of their product
  
  def multiply(M_1,M_2):
  #your code here
  
  #Write a function makeFunction(M) which takes a matrix and returns the linear map associated to the matrix.
  #For example, makeFunction([[1,2],[3,4]])([2,5]) = [12 32]
  
  def makeFunction(M):
  #your code here
  
  #Write a function makeMatrix(L) which takes a linear function L ( given as a python function which takes and returns lists) and returns the matrix associated to L.
  #For example if  
  #def L(v):
  #  return [2*v[0]+3*v[1], -4*v[0]]
  #Then makeMatrix(L) = BLAH (polymorphism is weird here)
  
  def makeMatrix(L):
  #your code here
  
  #You should double check for some examples that the following is true (as it ought to be)
  # makeMatrix(makeFunction(M_1)(makeFunction(M_2))) = multiply(M_1,M_2)
  
  
\end{python}

\end{document}
%%% Local Variables: 
%%% mode: latex
%%% TeX-master: t
%%% End: 
