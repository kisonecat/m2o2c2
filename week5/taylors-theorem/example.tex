\documentclass{ximera}

\title{An example}

\begin{document}

\begin{abstract}
  A specific example sheds light on Taylor series.
\end{abstract}\maketitle

Let's work through an example.

\begin{question}
  Let $f:\R^2 \to \R$ be a function.  All we know about $f$ at the point $(1,2)$ is the following:
  \begin{itemize}
  \item $f(1,2) = 6$
  \item \(Df(1,2) = \begin{bmatrix} 4 & 5 \end{bmatrix}\)
  \item \( D^2f(1,2)  = \begin{bmatrix} 1 & -2 \\ -2 & 3\end{bmatrix}\)
  \end{itemize} 
  
  Suppose that we want to approximate $f(1.1,1.9)$ as accurately as we can given this information.
  We can simply use the linear approximation to $f$ at $(1,2)$:
  
  \begin{solution}
    \begin{hint}
      \begin{align*}
        f(1.1,1.9 &\approx 6+\begin{bmatrix} 4 & 5 \end{bmatrix} \verticalvector{0.1\\-0.1}\\
        &=6+0.4-0.5\\
        &=5.9
      \end{align*}
    \end{hint}
    Using the linear approximation to $f$ at $(1,2)$, we find $f(1.1,1.9) \approx$ \answer{$5.9$}.
  \end{solution}
  
  This approximation ignores the second order data provided by the second derivative: we have essentially 
  assumed that the first derivative is constant along the line from $(1,2)$ to $(1.1,2.2)$.  Since we know the second 
  derivative at the point $(1,2)$ we can estimate how the derivative is changing along this line assuming the second 
  derivative was constant, and get a better approximation.
  
  For example, we could use the following three step process:
  
  \begin{itemize}
  \item Use the linear approximation to $f$ at $(1,2)$ to approximate $f(1.05,1.95)$
  \item Use the second derivative to approximate $Df(1.05,1.95)$
  \item Use the linear approximation to $f$ at $(1.05,1.95)$ to approximate $f(1.1,1.9)$
  \end{itemize}
  
  \begin{solution}
    \begin{hint}
      \begin{align*}
        f(1.05,1.95) &\approx 6+\begin{bmatrix} 4 & 5 \end{bmatrix} \verticalvector{0.05\\-0.05}\\
        &=6+0.2-0.25\\
        &=5.95
      \end{align*}
    \end{hint}
    Let's try that here:  $f(1.05,1.95) \approx$ \answer{$5.95$}
  \end{solution}

  \begin{solution}
    \begin{hint}
      \begin{align*}
        Df(1.05,1.95) &\approx Df(1,2)+\begin{bmatrix} 0.05 & -0.05\end{bmatrix}\begin{bmatrix} 1 & -2 \\ -2 & 3\end{bmatrix}\\
        &= \begin{bmatrix} 4 & 5 \end{bmatrix} + \begin{bmatrix} 0.05(1)+-0.05(-2) & 0.05(-2)+-0.05(3)\end{bmatrix}\\
        &=\begin{bmatrix} 4 & 5 \end{bmatrix} + \begin{bmatrix} 0.15 & -0.25\end{bmatrix}\\
        &=\begin{bmatrix} 4.15 & 4.75\end{bmatrix}
      \end{align*}
    \end{hint}
    Using the second derivative, $Df(1.05,1.95) $ is approximately:
    \begin{matrix-answer}
      correctMatrix = [['4.15','4.75']]
    \end{matrix-answer}
  \end{solution}

  \begin{solution}
    \begin{hint}
      \begin{align*}
        f(1.1,1.9) &\approx 5.95+\begin{bmatrix} 4.15 & 4.75 \end{bmatrix} \verticalvector{0.05\\-0.05}\\
        &=5.95+4.15(0.05)+4.75(-0.05)\\
        &=5.92
      \end{align*}
    \end{hint}
    Using the linear approximation to $f$ at $(1.05,1.95)$, $f(1.1,1.9) \approx$ \answer{$5.92$}
  \end{solution}
  
  So this method allowed us to get a slightly better approximation of $f(1.1,2.2)$ using the fact that the $Df_{\mathbf{p}}(\verticalvector{1\\-1})$ is increasing
  as $\mathbf{p}$ moves from $(1,2)$ in the direction $\verticalvector{1\\-1}$.  We really got a slightly higher estimate from $f(1.9,2.1)$ using this two step
  approximation compared to using the linear approximation because 
  $D^2f(1,2)\left(\verticalvector{1\\-1},\verticalvector{1\\-1} \right) = 8$ is positive.  
  
  We do not have to limit ourselves to only using a two step approximation:  we could get better and better approximations of $f(1.1,1.9)$
  by using more and more partitions of the line segment from $(1,2)$ to $(1.1,1.9)$.  For example, we could use ten partitions:
  \begin{itemize}
  \item Use the linear approximation to $f$ at $(1,2)$ to approximate $f(1.01,1.99)$
  \item Use the second derivative to approximate $Df(1.01,1.99)$
  \item Use the linear approximation to $f$ at $(1.01,1.99)$ to approximate $f(1.02,1.98)$
  \item Use the second derivative to approximate $Df(1.02,1.98)$
  \item Use the linear approximation to $f$ at $(1.02,1.98)$ to approximate $f(1.03,1.97)$
  \item $\vdots$
  \item  Use the linear approximation to $f$ at $(1.09,1.91)$ to approximate $f(1.1,1.9)$
  \end{itemize}
  
  This kind of process, where we are summing more and more of smaller and smaller values to approximate something, 
  furiously demands to be phrased as an integral.
  
  \begin{solution}
    \begin{hint}
      Notice that \begin{align*}
        D^2f\left(\verticalvector{0.1\frac{1}{n}\\-0.1\frac{1}{n}},\verticalvector{0.1\frac{1}{n}\\-0.1\frac{1}{n}}\right)
        &=\begin{bmatrix} 0.1 \frac{1}{n}, -0.1\frac{1}{n}\end{bmatrix}  \begin{bmatrix} 1 & -2 \\ -2 & 3\end{bmatrix} \verticalvector{0.1\frac{1}{n}\\-0.1\frac{1}{n}}\\
        &=\left(0.1(0.1)(1)+0.1(-0.1)(-2)+(0.1)(-0.1)(-2)+(-0.1)(-0.1)(3)\right)\frac{1}{n^2}\\
        &=0.08\frac{1}{n^2}
      \end{align*}
    \end{hint}
    \begin{hint}
      By partitioning $[0,1]$ into $n$ little pieces of equal width, the contribution to the sum 
      \begin{itemize}
      \item over $[0,\frac{1}{n}]$ is \(Df(1,2)\left( 0.1\frac{1}{n} \\ -0.1\frac{1}{n}\right) = \begin{bmatrix} 4 &5\end{bmatrix} \verticalvector{0.1\frac{1}{n} \\ -0.1\frac{1}{n}} = -0.1\frac{1}{n}\)
      \item over $[\frac{1}{n}, \frac{2}{n}]$ is 
        \begin{align*}
          Df(1+0.1\frac{1}{n},2+(-0.1)\frac{1}{n})\left( \verticalvector{0.1\frac{1}{n}\\(-0.1)\frac{1}{n}} \right) 
          &\approx Df(1,2)\left( \verticalvector{0.1\frac{1}{n}\\(-0.1)\frac{1}{n}}+D^2f\left(\right)\verticalvector{0.1\frac{1}{n}\\(-0.1)\frac{1}{n}},\verticalvector{0.1\frac{1}{n}\\(-0.1)\frac{1}{n}}\right)\\
          &= -0.1\frac{1}{n} + 0.08\frac{1}{n^2}
        \end{align*}
      \item over $[\frac{2}{n},\frac{3}{n}]$ is
        \begin{align*}
          Df(1+2(0.1\frac{1}{n}),2+2((-0.1)\frac{1}{n})) \left( \verticalvector{0.1\frac{1}{n}\\(-0.1)\frac{1}{n}} \right) 
          &\approx  Df(1+0.1\frac{1}{n},2+(-0.1)\frac{1}{n})\left( \verticalvector{0.1\frac{1}{n}\\(-0.1)\frac{1}{n}} \right) +D^2f\left(\verticalvector{0.1\frac{1}{n}\\(-0.1)\frac{1}{n}},\verticalvector{0.1\frac{1}{n}\\(-0.1)\frac{1}{n}}\right)\\
          &\approx -0.1\frac{1}{n} + 0.08\frac{1}{n^2}+ 0.08\frac{1}{n^2}\\
          &=1.4\frac{1}{n}+0.08\frac{2}{n}\frac{1}{n}
        \end{align*}
      \item $\vdots$
      \item over $[\frac{k+1}{n},\frac{k+2}{n}]$ is
        \begin{align*}
          Df(1+(k+1)(0.1\frac{1}{n}),2+(k+1)((-0.1)\frac{1}{n}))\left( \verticalvector{0.1\frac{1}{n}\\(-0.1)\frac{1}{n}} \right) 
          &\approx  Df(1+(k)0.1\frac{1}{n},2+(k)(-0.1)\frac{1}{n})\left( \verticalvector{0.1\frac{1}{n}\\(-0.1)\frac{1}{n}} \right) +D^2f\left(\verticalvector{0.1\frac{1}{n}\\(-0.1)\frac{1}{n}},\verticalvector{0.1\frac{1}{n}\\(-0.1)\frac{1}{n}}\right)\\
          &\approx -0.1\frac{1}{n} + (k-1)0.08\frac{1}{n^2}+ 0.08\frac{1}{n^2}\\
          &=1.4\frac{1}{n}+0.08\frac{k}{n}\frac{1}{n}
        \end{align*}
      \end{itemize}
      
    \end{hint}
    \begin{hint}
      So \(f(1.1,2.2) \approx 6+ \displaystyle\sum_{k=0}^{n} -0.1\frac{1}{n} +0.08\frac{k}{n}\frac{1}{n} \)
    \end{hint}
    \begin{hint}
      By definition of the integral we have \(\displaystyle\lim_{n \to \infty} \displaystyle\sum_{k=0}^{n} -0.1\frac{1}{n} +0.08\frac{k}{n}\frac{1}{n} = \displaystyle\int_0^1 (-0.1+0.08t) dt\)
    \end{hint}
    In this case, we get that $f(1.1,1.9) \approx 6+ \displaystyle\int_0^{1} f(t) dt$, where $f(t)=$\answer{$-0.1+0.08t$}
  \end{solution}
  
  \begin{solution}
    \begin{hint}
      \begin{align*}
        f(1.1,1.9) &\approx 6+ \displaystyle\int_0^1 (-0.1+0.08t) dt\\
        &=6+\left(-0.1t+\frac{1}{2}(0.08)t^2 \right)\big|_{0}^{1}\\
        &=6-0.1+0.04\\
        &=5.94
      \end{align*}
    \end{hint}
    Evaluating this integral we have $f(1.1,1.9) \approx $ \answer{$5.94$}.  This is the best approximation we can really expect to get given only this
    information about $f$ at $(1,2)$.
  \end{solution}
  
  Notice that this approximation of $f$ is really just 
  $f(1,2)+Df(1,2) \left(\verticalvector{0.1\\-0.1} \right)+ \frac{1}{2} D^2f(1,2)\left( \verticalvector{0.1\\-0.1}, \verticalvector{0.1\\-0.1}\right)$.
  
  The first two terms are just the regular linear approximation to $f$ at $(1,2)$, but the next term arose from integrating the function 
  $D^2f( \verticalvector{0.1\\-0.1},\verticalvector{0.1\\-0.1})t$ from $t=0$ to $t=1$.  This is  exactly
  $\frac{1}{2} D^2f(\verticalvector{0.1\\-0.1},\verticalvector{0.1\\-0.1})$.
  
  Generalizing, we might expect in general that 
  
  \begin{theorem}
    \(f(\mathbf{p} + \vec{h}) \approx f(\mathbf{p}) + Df(\mathbf{p})(\vec{h})+ \frac{1}{2} D^2f(\mathbf{p})\left( \vec{h},\vec{h}\right) \)
  \end{theorem}
  
  This is the second order taylor approximation of $f$ at $\mathbf{p}$.   Notice how similar it looks to the second order taylor approximation 
  of a single variable function!  If we had not taken the time to develop an understanding of $D^2f(\mathbf{p})$ as bilinear map, it would be quite messy 
  to even state this theorem, and it would only get worse for the higher order Taylor's theorem we will be learning about next week.
  
  Hopefully this (admittedly long) discussion has helped you to understand where this approximation comes from!  We will give a rigorous statement
  and proof of the theorem in the next section.
  
\end{question}

\end{document}
